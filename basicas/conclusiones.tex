\section{Conclusiones}

Se diseño una pala simplificada en dos dimensiones en forma de trapezoide y posteriormente se modeló en tres dimensiones. Obteniendo todos los valores necesarios para su inclusión en un estudio numérico mediante el programa MATLAB.\\

Fue realizado el estudio de las fuerzas relacionadas con el viento que afectaban a la pala de un aerogenerador para lograr el cálculo de un torque. Este torque fue definido para cada uno de los N segmentos separados del rotor una distancia basada en la longitud de la pala o radio. También se le aplicó un giro inicial a la pala para poder obtener energía. Esto se debe al mal aerodinamismo de la pala estudiada.\\

El principal elemento de estudio del trabajo se centraba en la torsión. La torsión de la pala es añadida al giro inicial o ángulo de cabeceo en busca de una mayor obtención de energía definida mediante la eficiencia. Se compararon sistemas en los cuales no se presentaba torsión frente a otros que sí, variando los que si tenían. Se obtuvieron resultados mejorados debido a la inclusión de la torsión en los cálculos de las simulaciones.\\

La simulación realizada expone diferentes problemas relacionadas con las hipótesis del diseño de la pala. La pala diseñada e introducida en la simulación se aleja de una manera muy obtusa de una bien diseñada. Otro se relaciona con el grosor de la pala en el borde de fuga cuando la mayoría se achatan, esto no se tuvo en cuenta e indujo a un peor diseño.\\

No se encontraron grandes diferencias entre un ángulo de torsión creciente o decreciente. Requiere estudio para poder determinar si hay alguna que sea mejor que la otra.\\

El número de segmentos también podría influir en una mejora de obtención de energía mediante torsión, requiere mayor profundidad de estudio.
